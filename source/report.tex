% Author: Ashwith Jerome Rego 
% Email: ashwith@ieee.org

\documentclass[12pt]{article}
\usepackage{cite}
\usepackage{color}
\usepackage{graphicx}
\usepackage{caption}
\usepackage{subcaption}
\usepackage{float}
\usepackage{amsmath}
\usepackage{url}


\title{The van Der Pol Oscillator}
\author{Ashwith Jerome Rego (143079021)}
\date{\today}

\begin{document}
\maketitle

\section{Introduction}
The van der Pol oscillator is an oscillator with nonlinear damping
governed by the following second-order differential equation
\eqref{van}
\cite{scholar}

\begin{equation}
\ddot{x} - \epsilon \left(1 - x^2\right)\dot{x} + x = 0
\label{van}
\end{equation}

In order to solve this numerically, it must be converted to a series
of first order ordinary differential equations (ODEs):

\begin{align}
y &= \dot{x}\\
\dot{y} &= \epsilon \left(1 - x^2\right)\dot{x} - x 
\end{align}

The following sections discuss the effect of varying the initial
conditions and the damping factor $\epsilon$


\section{Effect of Initial Conditions}

Figure \ref{fig:init} below shows the effect of varying the initial
conditions.

\begin{figure}[H]
  \centering
  \includegraphics[width=\textwidth]{../output/van_der_pol_init.eps}
  \caption{Effect of Initial Conditions}
  \label{fig:init}
\end{figure}

From the figure it can be seen that the initial conditions affect the
amplitude of the oscillations. This is expected since initial
conditions represent initial energy. It can also be seen that the
phase is affected. However, the frequency appears to be the same.

\section{Effect of Damping Factor ($\epsilon$)}

Figure \ref{fig:eps} shows how changing the damping factor affects
the oscillations.

\begin{figure}[H]
  \centering
  \includegraphics[width=\textwidth]{../output/van_der_pol_eps.eps}
  \caption{Effect of Damping Factor ($\epsilon$)}
  \label{fig:eps}
\end{figure}


It can be seen that when the damping factor is zero, the amplitude
varies sinusoidally. This is obvious when one notes that setting
$\epsilon = 0$ reduces \eqref{van} to:

\begin{equation}
\ddot{x} + x = 0
\end{equation}

which is the equation for simple harmonic motion. \cite{shm}.

As the damping factor is increased, there are two changes:

\begin{enumerate}
\item The frequency of oscillations decreases.
\item The waveforms appears to get closer to being a square wave.
\end{enumerate}

Changing the damping factor doesn't appear to change the amplitude of
oscillations with the exception of $\epsilon = 0$.

\bibliographystyle{ieeetr}
\bibliography{bibliography}
\end{document}


